\documentclass[12pt,letterpaper]{article}
\usepackage[letterpaper,margin=0.8in]{geometry}
\usepackage{enumitem}
\usepackage{hyperref}
\usepackage{marvosym}
\usepackage{xcolor}
%\usepackage{mdwlist}
\usepackage[T1]{fontenc}
%\usepackage{textcomp}
%\usepackage{tgpagella}
\definecolor{shade}{HTML}{701112}
\renewcommand{\rmdefault}{phv} % Arial
\renewcommand{\sfdefault}{phv}
\begin{document}
% If you want headings on subsequent pages,
% remove the ``%'' on the next line:
% \pagestyle{headings}

\begin{center}
    {\color{shade}\LARGE \textbf{Xiaoliang Wang}}\\
    \vspace{0.4em}
    Vadmyrveien 2, 5172 Loddefjord, Bergen, Norway \quad {\color{shade}\Large\Mobilefone} 45029792 \quad {\color{shade}\Large\Letter} wxlfrank@gmail.com
    \vspace{-0.8em}
\end{center}

\hrule
\vspace{5em}

\noindent 
June, 2015
\vspace{5em}

\noindent
Dear Sir/Madam who may be concerned,

I am writing to apply for the Java software developer position in your company which I saw in finn.no.
I am confident that my qualifications for the position merit your consideration.

As you can see from my resume, I will complete my PhD degree in software engineering from Bergen University this year.
%I decide to start industrial career rather than continuing my academic research when I graduate.
I know that the life in academics is different from the one in industry.
But the skills developed from academic research makes me competitive in the position.
\begin{itemize}
\item I have fundamental understanding of  software development process and techniques through long time study on software engineering
\item The academic research offers me the ability to analyse and find solutions to problems independently
\item Model Driven Engineering is a methodology to construct software from models by automatic model transformations. Model-driven engineering study enables me to analyse a problem from an abstract and high-view perspective
\item Writing and speaking skills are well trained through paper writing and discussion with colleagues
%\item Quick learning ability is developed during the study 
\end{itemize}
%I am PhD student learning computer science, specially in software engineering.
%My research area is model driven engineering which promote using model to construct software.
%When I learn this idea at the first time, I thought the idea is fascinating and amazing.
%As I dug the field deeper, I found that there is still big gap between reality and theory.
%Since software engineering is the discipline to guide software engineers to construct better software, 
%I would like to change to industrial field to put what I learn into use and learn new techniques from practice.
%
%I have stayed in school for long time, more than ten years.
%But I still have some experiences from practical work.
%During my master study, I implemented a tool to support model checking of asynchronous transition system.
%In addition, I also got an intern in Lenovo working on web page collection in back-end and display related data as front-end.
%Furthermore, after I graduated from master study, I was employed as a project staff to maintain a modelling tool and add requirement functionalities.
%There, I first learned model driven engineering. 
%For this work experience, I got the position of PhD student in Bergen University.
%During this period, I also maintain a modelling tool for my project.
%It should be mentioned that I am the main contributor of the tool for the last 3 years.
%
%Even though I have such practical experiences,  they are mainly in academic area.
%I know that there is big difference between academic tool development and industrial  software development.
%And I know that I am less experienced than the ones staying in industry for a long time.
%But I can learn new thing and like to learn from these experienced and skilled engineers.
%In addition, researching in software engineering give me a fundamental understanding of the technologies.
%I could quickly pick up the new technologies.

I enclosed my resume along with my application which thoroughly outlines my education and experiences. 
But I would appreciate the opportunity to demonstrate my skills in an interview.
%Please contact me at (+47)45029792.

Thank you so much for your consideration. Please do not hesitate to reach out to me with any questions.
I look forward to hearing from you.

\vspace{5mm}
\noindent
Sincerely,

\noindent
Wang, Xiaoliang

\end{document}
